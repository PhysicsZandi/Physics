\part{Logic and set theory}

\chapter{Logic}

    In this chapter, we will study propositional and predicate logic. Furthermore, we will introduce axiomatic systems.

\section{Propositional logic}

    \begin{definition}
        A proposition is a variable that can take true $T$ or false $F$ values.
    \end{definition}

    \noindent Logical operators build new propositions from old ones. There are four unary operators (Table~\ref{tab:unop}) and sixteen binary operators (Tables~\ref{tab:binop}~\ref{tab:binop2}~\ref{tab:binop3}). 

    \begin{table}[h!]
        \centering
        \begin{tabular}{c || c | c | c | c}
            $p$ & $\lnot p$ & $\id p$ & $\top p$ & $\bot p$ \\
            \hline
            \hline
            $T$ & $F$ & $T$ & $T$ & $F$ \\
            $F$ & $T$ & $F$ & $T$ & $F$ \\
        \end{tabular}
        \caption{All four possible unary operators: identity $\id$, negation $\lnot$, tautology $\top$, contradiction $\bot$.}
        \label{tab:unop}
    \end{table}

    \begin{table}[h!]
        \centering
        \begin{tabular}{c | c || c | c | c | c | c | c }
            $p$ & $q$ & $\id p$ & $\id q$ & $\neg p$ & $\neg q$ & $\top pq$ & $\bot qp$ \\
            \hline
            \hline
            $T$ & $T$ & $T$ & $T$ & $F$ & $F$ & $T$ & $F$ \\
            $T$ & $F$ & $T$ & $F$ & $F$ & $T$ & $T$ & $F$ \\
            $F$ & $T$ & $F$ & $T$ & $T$ & $F$ & $T$ & $F$ \\
            $F$ & $F$ & $F$ & $F$ & $T$ & $T$ & $T$ & $F$ \\
        \end{tabular}
        \caption{Six of all possible binary operators: identity $\id$, negation $\lnot$, tautology $\top$, contradiction $\bot$. }
        \label{tab:binop}
    \end{table}

    \begin{table}[h!]
        \centering
        \begin{tabular}{c | c || c | c | c | c | c | c }
            $p$ & $q$ & $p \land q$ & $\lnot (p \land q)$ & $p \lor q$ & $\lnot (p \lor q)$ & $p \veebar q$ & $\lnot (p \veebar q)$\\
            \hline
            \hline
            $T$ & $T$ & $T$ & $F$ & $T$ & $F$ & $F$ & $T$ \\
            $T$ & $F$ & $F$ & $T$ & $T$ & $F$ & $T$ & $F$ \\
            $F$ & $T$ & $F$ & $T$ & $T$ & $F$ & $T$ & $F$ \\
            $F$ & $F$ & $F$ & $T$ & $F$ & $T$ & $F$ & $T$ \\
        \end{tabular}
        \caption{Other six of all possible binary operators: and $\land$, or $\lor$, xor $\veebar$.}
        \label{tab:binop2}
    \end{table}

    \begin{table}[h!]
        \centering
        \begin{tabular}{c | c || c | c | c | c}
            $p$ & $q$ & $p \Rightarrow q$ & $\lnot (p \Rightarrow q)$ & $p \Leftarrow q$ & $\lnot (p \Leftarrow q)$\\
            \hline
            \hline
            $T$ & $T$ & $T$ & $F$ & $T$ & $F$\\
            $T$ & $F$ & $F$ & $T$ & $T$ & $F$\\
            $F$ & $T$ & $T$ & $F$ & $F$ & $T$\\
            $F$ & $F$ & $T$ & $F$ & $T$ & $F$\\
        \end{tabular}
        \caption{Remaining four of all possible binary operators: implications $\Rightarrow$ and $\Leftarrow$.}
        \label{tab:binop3}
    \end{table}

    \begin{theorem}
        All operators can be built from the nand operator.
    \end{theorem}

    \begin{theorem}
        Let $p$, $q$ be propositions. Then $p \Rightarrow q$ is equivalent to $\lnot p \Rightarrow \lnot q$.
    \end{theorem}
    \begin{proof}
        It can be simply proved by looking at the truth table:
        \begin{table}[h]
            \centering
            \begin{tabular}{c | c || c | c | c | c | c }
                $p$ & $q$ & $\lnot p$ & $\lnot q$ & $\lnot p \Rightarrow \lnot q$ & $p \Rightarrow q$ \\
                \hline
                \hline
                $T$ & $T$ & $F$ & $F$ & $T$ & $T$ \\
                $T$ & $F$ & $F$ & $T$ & $F$ & $F$ \\
                $F$ & $T$ & $T$ & $F$ & $T$ & $T$ \\
                $F$ & $F$ & $T$ & $T$ & $T$ & $T$ \\
            \end{tabular}
        \end{table}
    \end{proof}

    \begin{theorem}
        Let $p$, $q$ be propositions. Then $p \Leftrightarrow q$ is equivalent to $\lnot (p \veebar q)$.
    \end{theorem}
    \begin{proof}
        It can be simply proved by looking at the truth table:
        \begin{table}[h]
            \centering
            \begin{tabular}{c | c || c | c | c }
                $p$ & $q$ & $\lnot (p \veebar q)$ & $p \Leftrightarrow q$ \\
                \hline
                \hline
                $T$ & $T$ & $T$ & $T$ \\
                $T$ & $F$ & $F$ & $F$ \\
                $F$ & $T$ & $F$ & $F$ \\
                $F$ & $F$ & $T$ & $T$ \\
            \end{tabular}
        \end{table}
    \end{proof}

\section{Predicate logic}

    \begin{definition}
        A predicate $P(x)$ is a function of propositions. A predicate of two variables is called a relation.
    \end{definition}

    \begin{definition}
        Let $P(x)$ a predicate. Then the universal quantifier $\forall x \colon P(x)$ means that it is true if $P(x)$ is independent of $x$.
    \end{definition}

    \begin{definition}
        Let $P(x)$ a predicate. Then the existential quantifier $\exists x \colon P(x)$ is defined by 
        \begin{equation*}
            \exists x \colon P(x) \Leftrightarrow \lnot (\forall x \colon \lnot P(x)) ~.
        \end{equation*}
    \end{definition}

    \begin{definition}
        Let $P(x)$ a predicate. Then the unique existential quantifier $\exists ! x \colon P(x)$ is defined by
        \begin{equation*}
            \exists ! x \colon P(x) \Leftrightarrow \exists x \colon \forall y \colon ( P(y) \Leftrightarrow x = y ) ~.
        \end{equation*}
    \end{definition}

\section{Axiomatic systems}

    \begin{definition}
        A sequence of propositions (axioms) $a_1, \ldots a_N$ is called an axiomatic system.
    \end{definition}

    \begin{definition}
        Let $a_1, \ldots a_N$ be an axiomatic system. Then the proof of a proposition $p$ is a finite sequence of propositions $q_1, \ldots q_M=p$ such that either 
        \begin{enumerate}\label{proof}
            \item $q_i$ is an axiom, 
            \item $q_i$ a tautology or 
            \item $\exists n, m \in [1, i]$ such that $q_m \land q_n \Rightarrow q_i$ is true.
        \end{enumerate}
        In formula
        \begin{equation*}
            a_1, \ldots a_N \vdash p ~.
        \end{equation*}
    \end{definition}

    \noindent Propositional logic is an axiomatic system without axioms (all propositions are tautologies).

    \begin{definition}
        An axiomatic system is consistent if 
        \begin{equation*}
            \exists q \colon \lnot (a_1, \ldots a_N \vdash q) ~.
        \end{equation*}
    \end{definition}

    \noindent Intuitively, it means that there is no contradiction, since it is possible to prove anything from contradictions. 

    \begin{proof}
        Suppose there is a contradiction in the axioms $a_1, \ldots a, \lnot a , \ldots a_N$. Therefore, an arbitrary proposition $q$ can be proven by $a$, $\lnot a$ and $q$. The first two steps are axioms and the last follows from the tautology $(a \land \lnot a) \Rightarrow q$.
    \end{proof}

    \begin{theorem}
        Propositional logic is consistent.
    \end{theorem}
    \begin{proof}
        Propositional logic permits proving only tautologies. Indeed, there are no axioms, so in the proof only tautologies are allowed (even the third possibilities~\eqref{proof} imply tautologies because $q_m$ and $q_n$ are so). Therefore, given a proposition $q$, we cannot prove the contradiction $q \land \lnot q$.
    \end{proof}

    We can now state Godel's theorem.

    \begin{theorem}
        Any axiomatic system that contains elementary arithmetic is either inconsistent or there is a proposition that cannot be proven or disproven.
    \end{theorem}

    The continuum hypothesis (there is no set with cardinality between $\mathbb Z$ and $\mathbb R$) is the proposition for the Zermelo-Fraenkel axiomatic system.

\chapter{Set theory}

    In this chapter, we will study the ZFC (Zermelo-Fraenkel-Choice) axiomatic system. Suppose we have a relation $\epsilon$, called belonging, such that
    \begin{enumerate}
        \item $x \in y$ is equivalent to $\epsilon (x, y)$,
        \item $x \notin y$ is equivalent to $\lnot (x \in y)$,
        \item $x \subseteq y$ is equivalent to $\forall a \colon (a \in x \Rightarrow a \in y)$,
        \item $x = y$ is equivalent to $(x \subseteq y) \land (y \subseteq x)$,
        \item $x \subset y$ is equivalent to $(x \subseteq y) \land \lnot(y = x)$.
    \end{enumerate}

\section{Axiom of belonging relation}

    \begin{axiom}[Belonging relation]
        $x \in y$ is a proposition $\Leftrightarrow$ both $x, y$ are sets. In formula 
        \begin{equation*}
            \forall x \colon \forall y \colon (x \in y) \veebar \lnot (x \in y) ~.
        \end{equation*}
    \end{axiom}
    \noindent Intuitively, if $x \in y$ is not true, $x, y$ are not sets, because the xor in this case is a tautology (see Table~\eqref{tab:xor}).
    \begin{table}[h!]
        \centering
        \begin{tabular}{c | c || c }
            $p$ & $\lnot p$ & $p \veebar \lnot p$ \\
            \hline
            \hline
            $T$ & $F$ & $T$ \\
            $T$ & $F$ & $T$ \\
            $F$ & $T$ & $T$ \\
            $F$ & $T$ & $T$ \\
        \end{tabular}
        \caption{Xor binary operator for the first axiom, where $p = x \in y$. }
        \label{tab:xor}
    \end{table}

    The first axiom defines what is a set. To better understand this, consider the Russell's paradox: suppose we have a set that contains all the set that are not contained in themselves
    \begin{equation*}
        \exists u \colon \forall x \colon (x \notin x \Leftrightarrow x \in u) ~.
    \end{equation*}
    For reduction ad absurdum, suppose $u \in u$ is true. Therefore, $\lnot (u \notin u)$ is true and $u$ contains itself, but we have a contradiction
    \begin{equation*}
        u \in u \Rightarrow \lnot (u \in u) ~.
    \end{equation*} 
    On the other hand, suppose $u \notin u$ is true. Therefore, $\lnot (u \in u)$ is true and $u$ does not contain itself, but we have a contradiction 
    \begin{equation*}
        u \notin u \Rightarrow \lnot(u \notin u) ~.
    \end{equation*}
    Therefore, $u \in u$ is not a proposition and $u$ is not a set.

    Sometimes, the first axiom is replaced by the axiom of extensionality. 
    \begin{axiom}[Extensionality]
        Two sets are equal if they have the same elements. In formula 
        \begin{equation*}
            \forall x \colon \forall y \colon \forall z \colon (z \in x \Leftrightarrow z \in y) \Rightarrow x=y ~.
        \end{equation*}
    \end{axiom}

\section{Axiom of empty set}

    \begin{axiom}[Empty set]
        There exists an empty set. In formula 
        \begin{equation*}
            \exists y \colon \forall x \colon x \notin y ~.
        \end{equation*}
    \end{axiom}

    \begin{theorem}
        The empty set is unique.
    \end{theorem}
    \begin{proof}
        Suppose we have two empty sets $x$ and $x'$, in axioms
        \begin{equation*}
            a_1 \Leftrightarrow \forall y \colon u \notin x ~, \quad a_2 \Leftrightarrow \forall y \colon u \notin x' ~.
        \end{equation*}
        We begin with a tautology 
        \begin{equation*}
            q_1 \Leftrightarrow y \notin x \Rightarrow \forall y \colon (y \in x \Rightarrow u \in x') ~,
        \end{equation*}
        then using $a_1$, we have
        \begin{equation*}
            q_2 \Leftrightarrow \forall y \colon y \notin x ~,
        \end{equation*}
        and by the previous propositions $q_1$ and $q_2$ 
        \begin{equation*}
            q_3 \Leftrightarrow (\forall y \colon (y\in x \Rightarrow u \in x')) \Leftrightarrow x \subseteq x' ~.
        \end{equation*}
        Similarly, 
        \begin{equation*}
            q_4 \Leftrightarrow y \notin x' \Rightarrow \forall y \colon (y \in x' \Rightarrow u \in x) ~,
        \end{equation*}
        then using $a_1$, we have
        \begin{equation*}
            q_5 \Leftrightarrow \forall y \colon y \notin x' ~,
        \end{equation*}
        and by the previous propositions $q_4$ and $q_5$ 
        \begin{equation*}
            q_3 \Leftrightarrow (\forall y \colon (y\in x' \Rightarrow u \in x)) \Leftrightarrow x' \subseteq x ~.
        \end{equation*}
        Finally, using $q_3$ and $q_6$
        \begin{equation*}
            q_7 \Leftrightarrow ((x \subseteq x') \land (x' \subseteq x)) \Leftrightarrow x = x' ~.
        \end{equation*}
    \end{proof}

\section{Axiom of pair sets}

    \begin{axiom}[Pair sets]
        Let $x, y$ be sets. Then there exists the pair set $\{x,y\}$ containing $x$ and $y$. In formula
        \begin{equation*}
            \forall x \colon \forall y \colon \exists z \colon \forall u \colon (u \in z \Leftrightarrow (u = x \lor u = y)) ~.
        \end{equation*}
    \end{axiom}

    \begin{theorem}
        The pair is unordered, i.e.~$\{x,y\} = \{y, x\}$.
    \end{theorem}
    \begin{proof}
        In fact, by definition
        \begin{equation*}
            (a \in \{x, y\} \Rightarrow a \in \{y,x\}) \land (a \in \{y, x\} \Rightarrow a \in \{x,y\}) ~,
        \end{equation*}
        hence, 
        \begin{equation*}
            (\{x, y\} \subseteq \{y,x\}) \land (\{y, x\} \subseteq \{x,y\}) ~,
        \end{equation*}
        thus $\{x,y\} = \{y, x\}$.
    \end{proof}

    It is possible to define an ordered pair $(x,y) = \{x, \{x,y\}\}$ by 
    \begin{equation*}
        (x,y) = (a,b) \Leftrightarrow x = a \land y = b ~.
    \end{equation*}

    Furthermore, the single element set is $\{x\} = \{x, x\}$.

\section{Axiom of union set}

    \begin{axiom}[Union sets]
        Let $x, y$ be sets. Then there exists the union set $\cup x$ containing the elements of elements of $x$. In formula
        \begin{equation*}
            \forall x \colon \exists \cup x \colon \forall y \colon (y \in \cup x \Leftrightarrow \exists s \colon (y \in s \land s \in x)) ~.
        \end{equation*}
    \end{axiom}

    Given $x_1, \ldots x_N$ sets, we defined the union set to $x_{N+1}$ as 
    \begin{equation*}
        \{x_1, \ldots x_{N+1}\} = \bigcup \{\{a_1, \ldots a_N\},\{a_{N+1}\}\} ~.
    \end{equation*}
    It is a set by the pair set axiom. In the simple case of two sets $a, b$, we have $\cup x = \{a,b\}$.

    Notice that we can only take union of sets, so that the Russell's paradox is left out.

\section{Axiom of replacement}

    \begin{definition}
        A relation $f$ is a function if 
        \begin{equation*}
            \forall x \colon \exists! y \colon f(x,y) ~.
        \end{equation*}
    \end{definition}

    \begin{definition}
        The image of $f$ of a set $u$ consists of $y$ such that there is an $x \in u$ for which $f(x,y)$.
    \end{definition}

    \begin{axiom}[Replacement]
        Let $x$ be a set and $f$ a function. Then the image $\image_f (u)$ of $u$ under $f$ is a set.
    \end{axiom}

    It exists a weaker form of this axiom, called the principle of restricted comprehension. 
    \begin{theorem}
        Let $P(x)$ be a predicate and $u$ a set. Then the elements $y \in u$ such that $P(y)$ is true are a set $\{y \in u \colon P(y)\}$.
    \end{theorem}
    \begin{proof}
        Let us distinguish two cases. If $\lnot (\exists y \in u \colon P(y))$, then $\{y \in m \colon P(y)\} = \emptyset$. If $\exists \tilde y \in u \colon P(\tilde y)$, then we define $f(x,y) = (P(x) \land x = y) \lor (\lnot P(x) \land \tilde y = y )$ and $\{y \in m \colon P(y)\} = \image_f(u)$.
    \end{proof}
    The action of $P$ can be written as 
    \begin{enumerate}
        \item $\forall x y \colon \in P(x)$ is equivalent to $\forall x \colon (x \in y \Rightarrow P(x))$,  
        \item $\exists x \in y \colon \in P(x)$ is equivalent to $\lnot (\forall x \colon (x \in y \Rightarrow \lnot P(x)))$, or $\exists x \colon (x \in y \land P(x))$.
    \end{enumerate}

    \begin{definition}
        Let $x$ be a set. Then the intersection of $x$ is 
        \begin{equation*}
            \cap x = \{a \in \cup x \colon \forall b \in x \colon a \in b\} ~.
        \end{equation*}
        If $a, b in x$ and $\cap x = \emptyset$, then $a, b$ are disjoint.
    \end{definition}

    \begin{definition}
        Let $u, v$ be sets such that $u \subseteq v$. Then the complement of $u$ of $v$ is 
        \begin{equation*}
            u \textbackslash v = \{x \in v \colon x \notin u \} ~.
        \end{equation*}
    \end{definition}
    They are both sets by the axiom of replacement.

\section{Axiom of power sets}

    \begin{axiom}[Power sets]
        Let $u$ be a set. Then there exists a set $\mathcal P(u)$ composed by the subsets of $u$. In formula 
        \begin{equation*}
            \forall x \colon \exists y \colon \forall a \colon (a \in y \Leftrightarrow a \subseteq x) ~.
        \end{equation*}
    \end{axiom}

    \begin{definition}
        The Cartesian product of two sets $x, y$ is the set of all ordered pairs of elements of $x, y$
        \begin{equation*}
            x \times y \subseteq \mathcal P (\mathcal P ( \cup \{x,y\})) ~.
        \end{equation*}
    \end{definition}
    It is a set for the axioms of union, pair set, replacement and power set.

\section{Axiom of infinity}

    \begin{axiom}[Infinity]
        There exists a set that contains the empty set and the set itself. In formula
        \begin{equation*}
            \exists x \colon \emptyset \in x \land \forall y \colon (y \in x \Rightarrow \{y\} \in x) ~.
        \end{equation*}
    \end{axiom}

    \begin{theorem}
        $\mathbb N$ is a set.
    \end{theorem}
    \begin{proof}
        Suppose $x$ is such set. Therefore, $0 = \emptyset \in x$, $1 = \{\emptyset\} \in x$, $  = \{\{\emptyset\}\} \in x$ and so on.
    \end{proof}
    \noindent $\mathbb R = \mathcal P(\mathbb N)$ is also a set.

    Another version of this axiom is 
    \begin{axiom}[Infinity 2]
        There exists a set that contains the empty set and $y \cup \{y\} = \cup \{y, \{y\}\}$, where $y$ is one of its element.
    \end{axiom}
    Accordin with this axiom, we have $\mathbb N = \{\emptyset, \{\emptyset\}, \{\emptyset \{\emptyset\}\} \ldots \}$.

\section{Axiom of choice}

    \begin{axiom}[Choice]
        Let $x$ be a set such that it is not empty and its element are mutually disjoint. Then there exists a set $y$ containing exactly one element of each of $x$. In formula
        \begin{equation*}
            \forall x \colon P(x) \Rightarrow \exists y \colon \forall a \in x \colon \exists ! b \in a \colon a \in y ~,
        \end{equation*}
        where
        \begin{equation*}
            P(x) \Leftrightarrow (\exists a \colon a \in x ) \land (\forall a \colon \forall b \colon (a \in x \land b \in x) \Rightarrow \cap \{a,b\} = \emptyset) ~.
        \end{equation*}
    \end{axiom}

    The axiom of choice can be used to prove that every vector space has a basis and that there exists a complete system of representatives of an equivalence relation.

\section{Axiom of foundation}

    \begin{axiom}[Foundation]
        Every non-empty set $x$ contains an element $y$ such that it has no elements in common with $x$. In formula
        \begin{equation*}
            \forall x \colon (\exists a \colon a \in x) \Rightarrow \exists y \in x \colon \cap \{x, y\} = \emptyset ~.
        \end{equation*}
    \end{axiom}

\section{ZFC set theory}

    To summarise, the Zermelo-Fraenkel-Choice axiomatic systems for set theory is
    \begin{enumerate}
        \item belonging relation, i.e.~$\forall x \colon \forall y \colon (x \in y) \veebar \lnot (x \in y)$,
        \item empty set, i.e.~$\exists y \colon \forall x \colon x \notin y$,
        \item pair set, i.e.~$\forall x \colon \forall y \colon \exists z \colon \forall u \colon (u \in z \Leftrightarrow (u = x \lor u = y))$,
        \item union set, i.e.~$\forall x \colon \exists \cup x \colon \forall y \colon (y \in \cup x \Leftrightarrow \exists s \colon (y \in s \land s \in x))$,
        \item replacement, i.e.~the image $\image_f (u)$ of $u$ under $f$ is a set,
        \item power set, i.e.~$\forall x \colon \exists y \colon \forall a \colon (a \in y \Leftrightarrow a \subseteq x)$,
        \item infinity, i.e.~$\exists x \colon \emptyset \in x \land \forall y \colon (y \in x \Rightarrow \{y\} \in x)$,
        \item choice, i.e.~$\forall x \colon P(x) \Rightarrow \exists y \colon \forall a \in x \colon \exists ! b \in a \colon a \in y$, where $P(x) \Leftrightarrow (\exists a \colon a \in x ) \land (\forall a \colon \forall b \colon (a \in x \land b \in x) \Rightarrow \cap \{a,b\} = \emptyset)$,
        \item foundation, i.e.~$\forall x \colon (\exists a \colon a \in x) \Rightarrow \exists y \in x \colon \cap \{x, y\} = \emptyset$.
    \end{enumerate}

\chapter{Numbers}

    In this chapter, we will introduce the notion of map between sets and equivalence relations in order to find a construction of number sets: naturals, integers, rationals, reals.

\section{Maps}

    \begin{definition}
        Let $A, B$ be sets. Then a map $\phi\colon A \rightarrow B$ is a relation such that
        \begin{equation*}
            \forall a \in A \colon \exists! b \in B \colon \phi(a,b)~.
        \end{equation*}
        The set $A$ is called domain of $\phi$, $B$ is called target of $\phi$ and $\phi(A) = \image_\phi (A)$ is called the image of $A$ under $\phi$.
    \end{definition}

    \begin{definition}
        Let $\phi \colon A \rightarrow B$ be a map. Then $\phi$ is injective if 
        \begin{equation*}
            \forall x, y \in A \colon \phi(x) = \phi(y) \Rightarrow x = y ~,
        \end{equation*}
        surjective if 
        \begin{equation*}
            \image_\phi (A) = B ~,
        \end{equation*}
        bijective if it is both injective and surjective.
    \end{definition}

    \begin{definition}
        Two sets $A, B$ are isomorphic $A \simeq B$ if there exists a bijection $\phi \colon A \rightarrow B$.
    \end{definition}

    \begin{definition}
        Let $A$ be a set. Then it is infinite if 
        \begin{equation*}
            \exists B \subset A \colon B \simeq A ~,
        \end{equation*}
        in particular, it is countably infinite if $A \simeq \mathbb N$, otherwise uncountably infinite. It is finite if it is not infinite, where $A \simeq \{1, 2, \ldots N\}$ and $N$ is the cardinality of $A$.
    \end{definition}

    \begin{definition}
        The composition of two maps $\phi\colon A \rightarrow B$ and $\psi \colon B \rightarrow C$ is defined by
        \begin{equation*}
            \begin{tikzcd}[column sep=small]
                & A \arrow{dl}[swap]{\phi} \arrow[dr, "\phi \circ \psi"] & \\
                B \arrow{rr}[swap]{\phi} & & C
            \end{tikzcd}
        \end{equation*}
    \end{definition}

    \begin{theorem}
        Let $\phi \colon A \rightarrow B$, $\psi \colon B \rightarrow C$ and $\rho \colon C \rightarrow D$ maps. Then the composition of maps is associative.
    \end{theorem}
    \begin{proof}
        In fact,
        \begin{equation*}
            \begin{tikzcd}[column sep=huge, row sep=huge]
                A 
                \arrow{r}[yshift=0cm]{\phi} 
                \arrow{d}[swap]{\rho \circ \psi \circ \phi}
                \arrow{dr}[xshift=-1.2cm]{\psi \circ \phi}
                & B 
                \arrow{d}[xshift=0cm]{\psi} 
                \arrow{dl}[xshift=-1.2cm]{\rho \circ \psi} \\
                C 
                & D 
                \arrow{l}[xshift=0cm]{\rho}
            \end{tikzcd}
        \end{equation*}
    \end{proof}

    \begin{definition}
        Let $\phi \colon A \rightarrow B$ be a bijection. Then the inverse $\phi^{-1}$ of $\phi$ is defined by 
        \begin{equation*}
            \phi^{-1} \circ \phi = \id_A ~, \quad \phi \circ \phi^{-1} = \id_B ~.
        \end{equation*}
    \end{definition}

\section{Equivalence relation}

    \begin{definition}
        Let $A$ be a set and $\sim$ a relation, called an equivalence relation, such that it satisfies the following properties
        \begin{enumerate}
            \item reflexivity, i.e.~$\forall x \in A \colon x \sim x$,
            \item symmetry, i.e.~$\forall x,y \in A \colon x \sim y \Leftrightarrow y \sim x$,
            \item transitivity, i.e.~$\forall x,y,z \in A \colon (x \sim y \land y \sim z) \Rightarrow x \sim z$.
        \end{enumerate}
    \end{definition}

    \begin{definition}
        Let $\sim$ be an equivalence relation and $x \in A$. Then the equivalence class of $x$ is defined by 
        \begin{equation*}
            [x] = \{y \in A \colon x \sim y \} ~.
        \end{equation*}
    \end{definition}

    \begin{definition}
        Let $\sim$ be an equivalence relation. Then the quotient set of $A$ by $\sim$ is 
        \begin{equation*}
            A / \sim = \{[x] \in \mathcal P(A) \colon x \in A \} ~.
        \end{equation*}
    \end{definition}

    It is a set by the power set axiom. Due to the choice axiom, there exists a complete system of representatives, i.e.~a set $R \simeq M / \sim$.

    For example, we take $A = \mathbb Z$ and define 
    \begin{equation*}
        x \sim y \Leftrightarrow n - m \in 2 \mathbb Z ~.
    \end{equation*}
    Notice that 
    \begin{equation*}
        [0] = [2] = [-2] = \ldots ~, \quad [1] = [3] = [-1] = \ldots ~.
    \end{equation*}
    Therefore, $\mathbb Z / \sim = \{[0], [2]\}$. Furthermore, we define the addition 
    \begin{equation*}
        \oplus \colon \mathbb Z / \sim \times \mathbb Z \rightarrow \mathbb Z / \sim ~, \quad [x] \oplus [y] = [x+y] ~.
    \end{equation*}
    We need to check that it is independent by the choice 
    \begin{equation*}
        [x] \oplus [y] = [x'] \oplus [y'] ~. 
    \end{equation*}
    In fact, $[x] = [x']$ means $x-x' = 2n$ and $[y] = [y']$ means $y-y' = 2m$, with $n, m \in \mathbb Z$. Hence, 
    \begin{equation*}
        [x'+y'] = [x - 2n + y - 2m] = [x+y - 2 (n + m)] ~.
    \end{equation*}
    Now, by definition 
    \begin{equation*}
        x+y - 2 (n + m) - (x+y) = - 2 (n + m) \in 2 \mathbb Z ~,
    \end{equation*}
    implies that 
    \begin{equation*}
        [x'+y'] = [x+y] ~.
    \end{equation*}

\section{Number sets}

    